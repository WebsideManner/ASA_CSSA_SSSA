\documentclass{article}
\usepackage[english]{babel}
\usepackage{graphicx,capt-of}

%%%%%%%%%% Start TeXmacs macros
\newcommand{\tmtextbf}[1]{{\bfseries{#1}}}
\newcommand{\tmtextit}[1]{{\itshape{#1}}}
\newcommand{\tmtexttt}[1]{{\ttfamily{#1}}}
%%%%%%%%%% End TeXmacs macros

\begin{document}

\title{Temporal and Spatial Trends in Winter Wheat Yields Monitored Via
Multiple Online Data Sources}

\author{Peter Claussen}

\date{March 30, 2016}

\maketitle

\section{Introduction}

Recent studies have suggested that crop yields gains have slowed or stagnated
in many regions globally {\cite{lin.m-07-2012,graybosch.r-2014}}, while others
shows patterns of increasing yield {\cite{ray.d-12-2012}} . A simple analysis
of winter wheat yield in the central United States shows that while yield
gains have slowed for some (southern) regions, relative to mid-century gains,
yield improvement have accelerated in other (northern) regions (Figure
\ref{prelim}a). This implies a shift in winter wheat productions zones,
perhaps related to climate change.

To explore this shift, data from several public databases were combined for
covariate analysis to determine if changes in environmental and socio-economic
variables correlate with geospatial yield trends. For simplicity, this
analysis was centered on the Great Plains states plus adjacent states with
testing stations cooperating in the Hard Winter Wheat Regional Nursery
{\cite{hww-rpn}}. (Figure \ref{prelim}b). The period from 1984-2014 was chosen
by inspection of the inflection points in (Figure \ref{prelim}a).
\begin{center}
  \resizebox{0.85tex-line-width}{!}{\includegraphics{conference_paper_3-1.eps}}
  {\captionof{figure}{{\smaller}a. County average wheat yields (bu/acre) for
  six states in the Great Plains. b. Approximate distance (degrees
  latitude/longitude) from county center to nearest HWW RPN
  site.}}\label{prelim}
\end{center}

\section{Materials and Methods}

{\smaller}Data sources and brief descriptions:
\begin{itemize}
  \item USDA NASS {\cite{usda-nass}} Yield, acreage, income.
  
  \item NuGIS {\cite{nuGIS}} Soil $N$, $P$ , $K$ addition and withdrawal.
  
  \item CDC WONDER {\cite{modis,nldas}} Air and surface temperatures,
  precipitation.
  
  \item HWW RPN {\cite{hww-rpn}} Testing of advanced breeding lines.
\end{itemize}
For all data, linear regression coefficients on year $x_i$ and county $n$ were
computed in R using the model $y_{ni} = a_n + b_n x_i + e_{ni}$. Weighted
means were computed by by $\bar{y}_n = \hat{a}_n + \hat{b}_n \times 1999$.
Slopes were normalized to a relative rate as $\% \widehat{b_n} = 100 \times
\hat{b}_n / \bar{y}_n$.

\section{Spatial Plots, Winter Wheat Yield and Covariates}

\begin{center}
  \resizebox{0.99tex-line-width}{!}{\includegraphics{conference_paper_3-2.eps}}
  \resizebox{0.99tex-line-width}{!}{\includegraphics{conference_paper_3-3.eps}}
  {\captionof{figure}{a. Average winter wheat yields by county. b. Yearly
  percent change in county winter wheat yields}}
\end{center}
\begin{center}
  \resizebox{0.49tex-line-width}{!}{\includegraphics{conference_paper_3-4.eps}}
  \resizebox{0.49tex-line-width}{!}{\includegraphics{conference_paper_3-5.eps}}
  \resizebox{width=0.49tex-line-width}{!}{\includegraphics{conference_paper_3-6.eps}}
  \resizebox{width=0.49tex-line-width}{!}{\includegraphics{conference_paper_3-7.eps}}
\end{center}
\begin{center}
  \resizebox{0.24tex-line-width}{!}{\includegraphics{conference_paper_3-8.eps}}
  \resizebox{0.24tex-line-width}{!}{\includegraphics{conference_paper_3-9.eps}}
  \resizebox{0.24tex-line-width}{!}{\includegraphics{conference_paper_3-10.eps}}
  \resizebox{0.24tex-line-width}{!}{\includegraphics{conference_paper_3-11.eps}}
  {\captionof{figure}{Spatial distribution of changing covariates correlated
  with yield changes. Alphabetic identifiers correspond to Table
  \ref{coeftable}. Figures A,B,C,D,H and L represent county level changes over
  from 1984-2014. Figure M shows change in the average yields for entries in
  the HWW RPN. Figure N plots a spatial autocorrelation measure (Moran's I),
  and shows two areas where yield changes cluster. M and N are not included in
  the regression shown in Table \ref{coeftable}.}}
\end{center}

\section{Regression Coefficients}

The linear model relating change in yield ($Y_b$) to relative change in
covariates, $Y_b =\% \widehat{b_1} +\% \widehat{b_2} + \ldots +\%
\widehat{b_n}$, was fit in R using the \tmtexttt{lm} function. An optimal
linear model was found using the \tmtexttt{step} function. Table
\ref{coeftable} shows the result of this function. Coefficients listed are in
\% units, with the exception of Percent Harvested Acres, Wheat, which was
computed as the percent wheat acres harvested of total crop acres harvested.
\begin{center}
  \begin{tabular}{lrl}
    \hline
    \tmtextbf{Covariate} & \tmtextbf{Coefficient} & \\
    \hline
    A. Daily Max Air Temperature {\cite{nldas}} & -1.2558 & ***\\
    B. Daily Max Heat Index {\cite{nldas}} & 2.8354 & *\\
    C. Daily Precipitation {\cite{nldas}} & 0.2645 & ***\\
    D. Daily Sunlight {\cite{nldas}} & 0.7585 & \\
    E. Day Land Surface Temperature {\cite{modis}} & -0.0792 & *\\
    F. Farm Inputs, Tons $P$ {\cite{nuGIS}} & -0.0009 & \\
    G. Nutrient Removal by Crops, $N$ {\cite{nuGIS}} & -0.0114 & \\
    H. Net Removal to Use Ratio, $P_2 O_5$ {\cite{nuGIS}} & -0.0728 & ***\\
    I. Sum of 21 Crop Acres Planted {\cite{nuGIS}} & -0.1951 & *\\
    J. Sum of 21 Crop Acres Harvested {\cite{nuGIS}} & 0.3774 & ***\\
    K. Total Cropland {\cite{nuGIS}} & -0.1531 & ***\\
    L. Percent Harvested Acres, Wheat & -0.0122 & ***\\
    \hline
  \end{tabular} {\captionof{table}{Summary of best linear model coefficients.
  Coefficients of regression are significant at 0.1\% (***) 1\% (**) 5\% (*)
  and 10\%(.)}}\label{coeftable}
\end{center}

\section{Principal Components}

\begin{center}
  \resizebox{0.85tex-line-width}{!}{\includegraphics{conference_paper_3-12.eps}}
  {\captionof{figure}{Principle components of covariates associated with yield
  changes. See Table \ref{coeftable} for large letter codes. Small letter
  codes represent data points for counties, denoted by state.}}
\end{center}

\section{Conclusion}

{\smaller}This analysis demonstrates that multiple data sources can be
integrated to provide insight into regional yield trends. However,
difficulties managing heterogenous data, too numerous to list here, present a
challenge to updating this analysis as new data are available.

\begin{thebibliography}{1}
  \bibitem[1]{hww-rpn}Hard Winter Wheat Regional Nurseries
  \tmtexttt{www.ars.usda.gov/main/docs.htm?docid=11932}.
  
  \bibitem[2]{nuGIS}IPNI. 2012. A Nutrient Use Information System (NuGIS) for
  the U.S. Norcross, GA. \tmtexttt{www.ipni.net/nugis}.
  
  \bibitem[3]{modis}Moderate Resolution Imaging Spectroradiometer (MODIS)
  Daily Land Surface Temperature (LST), years 2003-2008 on CDC WONDER Online
  Database, released 2012. \tmtexttt{wonder.cdc.gov/nasa-lst.html}.
  
  \bibitem[4]{nldas}North America Land Data Assimilation System (NLDAS) Daily
  Air Temperatures and Heat Index, years 1979-2011 on CDC WONDER Online
  Database, released 2013. \tmtexttt{wonder.cdc.gov/NASA-NLDAS.html}.
  
  \bibitem[5]{usda-nass}United States Department of Agriculture, National
  Agricultural Statistics Service \tmtexttt{quickstats.nass.usda.gov}.
  
  \bibitem[6]{graybosch.r-2014}R~Graybosch, HE~Bockelman, KA~Garland-Campbell,
  DF~Garvin, and T~Regassa. {\newblock}\tmtextit{Yield Gains in Major U.S.
  Field Crops}, chapter Wheat. {\newblock}ASA, CSSA and SSSA, 5585 Guilford
  Rd. Madison, WI 53711-5801, USA, September 2014.
  
  \bibitem[7]{lin.m-07-2012}M~Lin and P~Huybers. {\newblock}Reckoning wheat
  yield trends. {\newblock}\tmtextit{Environ Res Lett}, 7, July 2012.
  
  \bibitem[8]{ray.d-12-2012}DK~Ray, N~Ramankutty, ND~Mueller, PC~West, and
  JA~Foley. {\newblock}Recent patterns of crop yield growth and stagnation.
  {\newblock}\tmtextit{Nature communications}, 3:1293, December 2012.
\end{thebibliography}

\end{document}
