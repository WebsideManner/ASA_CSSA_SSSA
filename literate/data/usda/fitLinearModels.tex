\documentclass{report}
\usepackage{amsmath}

\usepackage{Sweave}
\begin{document}
We start our analysis centered on the USDA NAS data set at \verb|http://quickstats.nass.usda.gov|. The parameters for this query is

   \begin{verbatim}
   Program : SURVEY
   Sector : CROPS
   Group : FIELD CROPS
   Commodity : WHEAT
   Category : YIELD
   Data Item : WHEAT, WINTER - YIELD, MEASURED in BU / ACRE
   Domain : TOTAL
   Location : COUNTY
   State : KANSAS, NEBRASKA, NORTH DAKOTA, OKLAHOMA, SOUTH DAKOTA, TEXAS
   Period Type: ANNUAL
   Period: YEAR
   \end{verbatim}

   Note that there are data for insect damage, etc.

   all available years
   36000+ records


This document drives the final analysis, so we save a template for covariates

Trial maps use lower case state names, so we manipulate to create a single state-county index.

We note that there are a few counties in Texas that have yield values of O. I'm not sure crop failures are part of what we want to measure, so I'm excluding these.
\begin{Schunk}
\begin{Sinput}
> yield.field <- "WHEAT, WINTER - YIELD, MEASURED IN BU / ACRE"
> production.field <- "WHEAT, WINTER - PRODUCTION, MEASURED IN BU"
> if(!file.exists("county.yield.dat.Rda")) {
+    county.yield.dat <- read.csv("69C9E26C-F285-325A-9443-6668747878F8.csv", header = TRUE)
+    
+    county.yield.dat <- subset(county.yield.dat,county.yield.dat$Value>1)
+    county.yield.dat <- subset(county.yield.dat, county.yield.dat$County != "OTHER (COMBINED) COUNTIES")
+    county.yield.dat$statecounty <- create.index(county.yield.dat)
+    county.yield.dat$County <- as.factor(county.yield.dat$County)
+    county.yield.dat$State <- as.factor(county.yield.dat$State)
+ 
+    county.early.dat <- subset(county.yield.dat,early.f(county.yield.dat$Year))
+    county.mid.dat <- subset(county.yield.dat,mid.f(county.yield.dat$Year))
+    county.late.dat <- subset(county.yield.dat,late.f(county.yield.dat$Year))
+    
+    save(county.early.dat, file="county.early.dat.Rda")
+    save(county.mid.dat, file="county.mid.dat.Rda")
+    save(county.late.dat, file="county.late.dat.Rda")
+    
+    save(county.yield.dat, file="county.yield.dat.Rda")
+ }
> if(!file.exists("full.yield.dat.Rda")) {
+    
+    county.yield1.dat <- read.csv("69C9E26C-F285-325A-9443-6668747878F8.csv", header = TRUE)
+    county.yield2.dat <- read.csv("CC71A450-97E2-3D03-BB42-39ACFAFBE2E6.csv", header = TRUE)
+    full.yield.dat <- rbind(county.yield1.dat,county.yield2.dat, model=FALSE, qr=keepqr)
+    county.yield1.dat=NULL
+    county.yield2.dat=NULL
+    full.yield.dat <- subset(full.yield.dat,county.yield.dat$Value>1)
+    full.yield.dat <- subset(full.yield.dat, full.yield.dat$County != "OTHER (COMBINED) COUNTIES")
+    full.yield.dat$statecounty <- create.index(full.yield.dat)
+    full.yield.dat$County <- as.factor(full.yield.dat$County)
+    full.yield.dat$State <- as.factor(full.yield.dat$State)
+ 
+    full.early.dat <- subset(full.yield.dat,early.f(full.yield.dat$Year))
+    full.mid.dat <- subset(full.yield.dat,mid.f(full.yield.dat$Year))
+    full.late.dat <- subset(full.yield.dat,late.f(full.yield.dat$Year))
+    
+    save(full.early.dat, file="full.early.dat.Rda")
+    save(full.mid.dat, file="full.mid.dat.Rda")
+    save(full.late.dat, file="full.late.dat.Rda")
+    
+    save(full.yield.dat, file="full.yield.dat.Rda")
+ }
\end{Sinput}
\end{Schunk}


We do not have yield estimates for every county, for every year. We set a minimum number of yield values to be used 
to determine yield change.
\begin{Schunk}
\begin{Sinput}
> #min.count <- 10
\end{Sinput}
\end{Schunk}

I've decided to compare early, mid and late century trends. I'll use functions to simplify this, so I might more easily shift ranges later. We'll use the most recent 30 years (2014-1984).
Base on inspection of the state level curves, the inflection point is about 1978 and the upswing starts in the early 40s.

\begin{Schunk}
\begin{Sinput}
> if(!file.exists("late.county.lm.Rda")) {
+    load.if.needed("county.late.dat")
+    load.if.needed("county.mid.dat")
+    load.if.needed("county.early.dat")
+    late.counts <- tapply(county.late.dat$Value,list(county.late.dat$statecounty),length)
+    county.late.dat$Count <- late.counts[county.late.dat$statecounty]
+    county.late.dat <- subset(county.late.dat,county.late.dat$Count>(min.count-1))
+    
+    late.county.lm <- lm(Value ~ 0 + statecounty + statecounty:Year,data=county.late.dat, model=FALSE, qr=keepqr)  
+    
+    mid.counts <- tapply(county.mid.dat$Value,list(county.mid.dat$statecounty),length)
+    county.mid.dat$Count <- mid.counts[county.mid.dat$statecounty]
+    county.mid.dat <- subset(county.mid.dat,county.mid.dat$Count>(min.count-1))
+    mid.lm <- lm(Value ~ 0 + statecounty + statecounty:Year,data=county.mid.dat, model=FALSE, qr=keepqr)
+    #summary(mid.lm)
+ 
+    early.counts <- tapply(county.early.dat$Value,list(county.early.dat$statecounty),length)
+    county.early.dat$Count <- early.counts[county.early.dat$statecounty]
+    county.early.dat <- subset(county.early.dat,county.early.dat$Count>(min.count-1))
+    early.lm <- lm(Value ~ 0 + statecounty + statecounty:Year,data=county.early.dat, model=FALSE, qr=keepqr)
+    
+    save(late.county.lm, file="late.county.lm.Rda")
+    save(early.lm, file="early.lm.Rda")
+    save(mid.lm, file="mid.lm.Rda")
+    
+    #summary(lm(Value ~ Year,data=subset(county.late.dat,county.late.dat$statecounty=="kansas,allen")))
+    #summary(lm(Value ~ Year,data=subset(county.late.dat,county.late.dat$statecounty=="kansas,anderson")))
+    #summary(lm(Value ~ Year,data=subset(county.late.dat,county.late.dat$statecounty=="kansas,atchison")))
+ }
> if(!file.exists("full.county.lm.Rda")) {
+    load.if.needed("full.late.dat")
+    load.if.needed("full.mid.dat")
+    load.if.needed("full.early.dat")
+    late.counts <- tapply(full.late.dat$Value,list(full.late.dat$statecounty),length)
+    full.late.dat$Count <- late.counts[full.late.dat$statecounty]
+    full.late.dat <- subset(full.late.dat,full.late.dat$Count>(min.count-1))
+ 
+    late.full.lm <- lm(Value ~ 0 + statecounty + statecounty:Year,data=full.late.dat, model=FALSE, qr=keepqr)
+    
+    mid.counts <- tapply(full.mid.dat$Value,list(full.mid.dat$statecounty),length)
+    full.mid.dat$Count <- mid.counts[full.mid.dat$statecounty]
+    full.mid.dat <- subset(full.mid.dat,full.mid.dat$Count>(min.count-1))
+    mid.full.lm <- lm(Value ~ 0 + statecounty + statecounty:Year,data=full.mid.dat, model=FALSE, qr=keepqr)
+    #summary(mid.lm)
+ 
+    early.counts <- tapply(full.early.dat$Value,list(full.early.dat$statecounty),length)
+    full.early.dat$Count <- early.counts[full.early.dat$statecounty]
+    full.early.dat <- subset(full.early.dat,full.early.dat$Count>(min.count-1))
+    early.full.lm <- lm(Value ~ 0 + statecounty + statecounty:Year,data=full.early.dat, model=FALSE, qr=keepqr)
+    
+    save(late.full.lm, file="late.full.lm.Rda")
+    save(early.full.lm, file="early.full.lm.Rda")
+    save(mid.full.lm, file="mid.full.lm.Rda")
+ }
\end{Sinput}
\end{Schunk}

\begin{Schunk}
\begin{Sinput}
> load.if.needed("county.early.dat")
> load.if.needed("county.mid.dat")
> load.if.needed("county.late.dat")
> grid.arrange(arrangeGrob(
+   ggplot(county.early.dat, aes(Year,Value)) + geom_smooth(aes(group=State,color=State),weight=10,se = FALSE,method="lm"),
+   ggplot(county.mid.dat, aes(Year,Value)) + geom_smooth(aes(group=State,color=State),weight=10,se = FALSE,method="lm"),
+   ggplot(county.late.dat, aes(Year,Value)) + geom_smooth(aes(group=State,color=State),weight=10,se = FALSE,method="lm"),
+    as.table=TRUE,
+    ncol=1))
\end{Sinput}
\end{Schunk}
\includegraphics{fitLinearModels-005}

\begin{Schunk}
\begin{Sinput}
> load.if.needed("county.yield.dat")
> county.yield.plot <- ggplot(county.yield.dat, aes(Year,Value)) 
> county.yield.plot <- county.yield.plot + scale_colour_brewer(palette="Set1") 
> #county.yield.plot <- county.yield.plot + geom_point(aes(color=Ag.District),size=2,alpha = 0.3)
> county.yield.plot <- county.yield.plot + geom_point(aes(color=State),size=1,alpha = 0.3)
> county.yield.plot <- county.yield.plot  + geom_smooth(aes(group= State,color=State),weight=10,se = FALSE,method="loess",span = 0.7)
> county.yield.plot
\end{Sinput}
\end{Schunk}
\includegraphics{fitLinearModels-006}

\section{Create covariate base}
We generate a base file for later covariate analysis of county level wheat yields. We start with yield regression.
\begin{Schunk}
\begin{Sinput}
> load.if.needed("late.county.lm")
> load.if.needed("county.late.dat")
\end{Sinput}
\end{Schunk}

\includegraphics{fitLinearModels-008}


\begin{Schunk}
\begin{Sinput}
> yield.a.estimates <- extract.county.estimates(late.county.lm,term=1)
> yield.b.estimates <- extract.county.estimates(late.county.lm,term=2)
> if(!file.exists("covariates.dat.Rda")) {
+    sum(names(yield.a.estimates)!=names(yield.b.estimates))
+    statecounty=names(yield.a.estimates)
+    state <- statecounty
+    for(i in 1:length(state)) {
+       current <- state[i]
+       tmp <- strsplit(current,",")
+       current <- tolower(tmp[[1]][1])
+       state[i] <- current
+    }
+    covariates.dat <- data.frame(
+       statecounty=statecounty,
+       state=state,
+       yield.a=yield.a.estimates,
+       yield.b=yield.b.estimates
+       )
+    save(covariates.dat, file="covariates.dat.Rda")
+ }
\end{Sinput}
\end{Schunk}

\includegraphics{fitLinearModels-010}

\begin{Schunk}
\begin{Sinput}
> yield.a.full.estimates <- extract.county.estimates(late.full.lm,term=1)
> yield.b.full.estimates <- extract.county.estimates(late.full.lm,term=2)
> if(!file.exists("covariates.full.dat.Rda")) {
+    sum(names(yield.a.full.estimates)!=names(yield.b.full.estimates))
+    statecounty=names(yield.a.full.estimates)
+    state <- statecounty
+    for(i in 1:length(state)) {
+       current <- state[i]
+       tmp <- strsplit(current,",")
+       current <- tolower(tmp[[1]][1])
+       state[i] <- current
+    }
+    full.covariates.dat <- data.frame(
+       statecounty=statecounty,
+       state=state,
+       yield.a=yield.a.full.estimates,
+       yield.b=yield.b.full.estimates
+       )
+    save(full.covariates.dat, file="full.covariates.dat.Rda")
+ }
\end{Sinput}
\end{Schunk}

\includegraphics{fitLinearModels-012}

\includegraphics{fitLinearModels-013}

\includegraphics{fitLinearModels-014}

\begin{Schunk}
\begin{Sinput}
> names(values)[!names(values)%in%names(estimate.mean)]
\end{Sinput}
\begin{Soutput}
 [1] "nebraska,blaine"        "nebraska,dixon"         "nebraska,grant"        
 [4] "nebraska,thomas"        "nebraska,wayne"         "nebraska,wheeler"      
 [7] "oklahoma,latimer"       "oklahoma,pushmataha"    "south dakota,lincoln"  
[10] "south dakota,minnehaha" "texas,angelina"         "texas,austin"          
[13] "texas,bandera"          "texas,blanco"           "texas,calhoun"         
[16] "texas,camp"             "texas,chambers"         "texas,crockett"        
[19] "texas,culberson"        "texas,dimmit"           "texas,ector"           
[22] "texas,edwards"          "texas,fort bend"        "texas,goliad"          
[25] "texas,grimes"           "texas,hardin"           "texas,harrison"        
[28] "texas,hidalgo"          "texas,hudspeth"         "texas,jackson"         
[31] "texas,jeff davis"       "texas,jefferson"        "texas,kerr"            
[34] "texas,kleberg"          "texas,marion"           "texas,matagorda"       
[37] "texas,maverick"         "texas,morris"           "texas,nacogdoches"     
[40] "texas,real"             "texas,refugio"          "texas,rusk"            
[43] "texas,san patricio"     "texas,shelby"           "texas,somervell"       
[46] "texas,titus"            "texas,tyler"            "texas,victoria"        
[49] "texas,waller"           "texas,willacy"          "texas,winkler"         
\end{Soutput}
\begin{Sinput}
> names(estimate.mean)[!names(estimate.mean)%in%names(values)]
\end{Sinput}
\begin{Soutput}
character(0)
\end{Soutput}
\end{Schunk}

